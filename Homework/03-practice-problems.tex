\documentclass[a4paper,10pt,landscape,twocolumn]{scrartcl}

%% Settings
\newcommand\problemset{2}
\newcommand\deadline{Friday November 18th, 20:00h}
\newif\ifcomments
\commentsfalse % hide comments
%\commentstrue % show comments

% Packages
\usepackage[english]{exercises}
\usepackage{wasysym}
\usepackage{hyperref}
\hypersetup{colorlinks=true, urlcolor = blue, linkcolor = blue}

\begin{document}

\practiceproblems

{\sffamily\noindent
%This week's exercises deal with sets, counting and uniform probabilities.
This week's exercises deal with Huffman codes, arithmetic codes and the AEP. You do not have to hand in these exercises, they are for practicing only. Problems marked with a $\bigstar$ are generally a bit harder. If you have questions about any of the exercises, please post them in the \href{https://www.moodle.ch/lms/mod/forum/view.php?id=1761}{discussion forum on Moodle}, and try to help each other. We will also keep an eye on the forum.
}

\begin{exercise}[Non-prefix-free arithmetic codes]
In class, we have seen a procedure to build a prefix-free arithmetic code $AC$ for $X$ by dividing $[0,1)$ into smaller intervals $I_x$ (for $x \in \mathcal{X}$) according to the probability distribution $P_X$, and picking $AC(x)$ to be the (name of the) largest binary interval that fits into $I_x$. In this exercise, we consider a simpler procedure that creates slightly shorter codewords, but is not necessarily prefix-free.
	\begin{subex}
	Given $X$ with $\mathcal{X} = \{\mathtt{a,b,c,d}\}$ and $P_X(\mathsf{a}) = P_X(\mathsf{b}) = 1/3$, $P_X(\mathsf{c}) = P_X(\mathsf{d}) = 1/6$. Draw the intervals $I_x$ on $[0,1)$. Then assign codewords to each $x$ by finding a number in each interval with a binary representation that is as short as possible. Note that there are sometimes multiple possibilities!
	\end{subex}
	\begin{subex}
	Also find the prefix-free arithmetic code $AC^{pf}$ for this source. How do the average codeword lengths compare?
	\end{subex}
	\begin{subex}
	Recall the proof that $AC^{pf}(X)
	\end{subex}
\end{exercise}











\end{document}