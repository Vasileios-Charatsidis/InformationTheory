\documentclass[a4paper,10pt,landscape,twocolumn]{scrartcl}

%% Settings
\newcommand\problemset{5}
\newcommand\deadline{Friday November 18th, 20:00h}
\newif\ifcomments
\commentsfalse % hide comments
%\commentstrue % show comments

% Packages
\usepackage[english]{exercises}
\usepackage{wasysym}
\usepackage{hyperref}
\hypersetup{colorlinks=true, urlcolor = blue, linkcolor = blue}

\begin{document}

\practiceproblems

{\sffamily\noindent
This week's exercises deal with random walks, noisy channels and error-correcting codes. You do not have to hand in these exercises, they are for practicing only. Problems marked with a $\bigstar$ are generally a bit harder. If you have questions about any of the exercises, please post them in the \href{https://www.moodle.ch/lms/mod/forum/view.php?id=1761}{discussion forum on Moodle}, and try to help each other. We will also keep an eye on the forum.
}

\begin{exercise}[Random walk on a chessboard]
Consider a 4x4 chessboard. We let a knight (who can move 2 spaces horizontally and 1 vertically or 1 space horizontally and 2 vertically) perform a random walk on this chessboard, choosing his move uniformly random every time. What is the entropy rate of this process?
\end{exercise}

\begin{exercise}[Repetition code]
Consider the repetition code $R_9$. One way of viewing this code is as a \emph{concatenation} of $R_3$ with itself. We first encode the source stream with $R_3$, then encode the resulting output with $R_3$ again. We could call this code $R_3^2$. This idea motivates an alternative decoding algorithm, in which we decode the bits three at a time using the decoder for $R_3$, and then decode the decoded bits from that first decoder using the decoder for $R_3$.

Evaluate the probability of error for this decoder and compare it with the probability of error for the optimal decoder for $R_9$.

Can you think of reasons to use $R_3^2$ (instead of $R_9$) in practice?
\end{exercise}

\begin{exercise}[Another linear code]
Consider the following linear code $C$ given by the generator matrix
\[
G^T = \left(
\begin{array}{c c c}
1&0&0\\
0&1&0\\
0&0&1\\
1&1&0\\
1&0&1\\
0&1&1\\
1&1&1
\end{array}
\right)
\]

\begin{subex}
Find the parity check matrix $H$.
\end{subex}

\begin{subex}
How many bits can $C$ encode? How long are its codewords? How many different codewords are there?
\end{subex}

\begin{subex}
What is the minimal distance?
\end{subex}

\begin{subex}
Encode the strings 101, 111 according to $C$.
\end{subex}

\begin{subex}
Decode 1011010, 1110110, 1111110, and 1111111.
\end{subex}
\end{exercise}














\end{document}