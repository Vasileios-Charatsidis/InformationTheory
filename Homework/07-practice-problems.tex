\documentclass[a4paper,10pt,landscape,twocolumn]{scrartcl}

%% Settings
\newcommand\problemset{7}
\newcommand\deadline{Friday November 18th, 20:00h}
\newif\ifcomments
\commentsfalse % hide comments
%\commentstrue % show comments

% Packages
\usepackage[english]{exercises}
\usepackage{wasysym}
\usepackage{hyperref}
\hypersetup{colorlinks=true, urlcolor = blue, linkcolor = blue}

\usepackage{tikz}

\begin{document}

\practiceproblems

{\sffamily\noindent
This week's exercises deal with noisy-channel coding. You do not have to hand in these exercises, they are for practicing only. Problems marked with a $\bigstar$ are generally a bit harder. If you have questions about any of the exercises, please post them in the \href{https://www.moodle.ch/lms/mod/forum/view.php?id=2219}{discussion forum on Moodle}, and try to help each other. We will also keep an eye on the forum.
}

\begin{exercise}[Encoder and decoder as part of the channel]
% [CT, 7.16] 
Consider a binary symmetric channel with crossover probability 0.1. A possible coding scheme for this channel with two codewords of length 3 is to encode message $w_1$ as 000 and $w_2$ as 111. Decoding happens by majority vote. With this coding scheme, we can consider the combination of encoder, channel, and decoder as forming a new BSC, with two inputs $w_1$ and $w_2$, and two outputs $w_1$ and $w_2$.

\begin{subex}
Draw the channel and calculate the crossover probability.
\end{subex}

\begin{subex}
What is the capacity of the original channel?
\end{subex}

\begin{subex}
What is the capacity of this new channel in bits per transmission of the original channel? Compare.
\end{subex}

\begin{subex}
Prove the general statement that for any channel, considering the encoder, channel, and decoder together as a new channel from messages to estimated messages will not increase the capacity in bits per transmission of the original channel.
\end{subex}

\end{exercise}

\begin{exercise}[Source and channel]
% [CT, 7.31] 
We wish to encode a Bernoulli($\alpha$) process $V_1, V_2, ...$ for transmission over a binary symmetric channel with crossover probability $\epsilon$.
\begin{center}
\begin{tikzpicture}
\node at (0,0) {$V^n$};
\node at (2,0) {$X^n(V^n)$};
\node at (4,0) {BSC($\epsilon$)};
\node at (6,0) {$Y^n$};
\node at (8,0) {$\hat{V}^n$};
\draw[->, >=latex] (0.5,0) -- (1.25,0);
\draw[->, >=latex] (2.75,0) -- (3.5,0);
\draw[->, >=latex] (4.75,0) -- (5.5,0);
\draw[->, >=latex] (6.5,0) -- (7.5,0);
\end{tikzpicture}
\end{center}
Find conditions on $(\alpha,\epsilon)$ under which the error probability $P[\hat{V}^n \neq V^n]$ can be made to go to zero as $n \to \infty$.
\end{exercise}

\begin{exercise}[Channel with memory]
% [CT, 7.36] 
Consider the discrete memoryless channel $(\mathcal{X}, P_{Y|X}, \mathcal{Y})$ with $\mathcal{X} = \{-1,1\}$, and $Y = ZX$ for a random variable $Z$ with $\mathcal{Z} = \{-1,1\}$.
\begin{subex}
What is the capacity of this channel when $Z$ is uniform?
\end{subex}
\begin{subex}
Now consider the channel with memory. Before transmission begins, $Z$ is randomly chosen and fixed for all time. What is the capacity when $Z$ is uniform?
\end{subex}
\end{exercise}

\begin{exercise}[Practice]
If you have solved all of the above problems before the exercise session is over, take the rest of the time to review previous exercise sets, homeworks or concept quizzes, or to ask any questions that you might have. When done, take on the \href{https://www.moodle.ch/lms/mod/forum/view.php?id=1788}{challenges}.
\end{exercise}













\end{document}