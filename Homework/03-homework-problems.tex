\documentclass[a4paper,10pt,landscape,twocolumn]{scrartcl}

%% Settings
\newcommand\problemset{3}
\newcommand\deadline{Friday November 25th, 20:00h}
\newif\ifcomments
\commentsfalse % hide comments
%\commentstrue % show comments

% Packages
\usepackage[english]{exercises}
\usepackage{wasysym}
\usepackage{hyperref}
\hypersetup{colorlinks=true, urlcolor = blue, linkcolor = blue}

\usepackage{tikz}

\begin{document}

\homeworkproblems

{\sffamily\noindent
Your homework must be handed in \textbf{electronically via Moodle before \deadline}. This deadline is strict and late submissions are graded with a 0. At the end of the course, the lowest of your 6 weekly homework grades will be dropped. You are strongly encouraged to work together on the exercises, including the homework. However, after this discussion phase, you have to write down and submit your own individual solution. Numbers alone are never sufficient, always motivate your answers.
}

\begin{exercise}[Shannon code (6pt)]
Consider the following method for generating a code for a random variable $X$ which takes on $m$ values $\{1,2,...,m\}$. Assume that the probabilities are ordered such that $P_X(1) \geq P_X(2) \geq ... \geq P_X(m)$. Define
\[
F_i := \sum_{k=1}^{i-1} P_X(k),
\]
the sum of the probabilities of all symbols less than $i$. Then the Shannon code is defined by assigning the (binary representation of the) number $F_i \in [0,1]$ as the codeword for $i$, where $F_i$ is rounded off to $\lceil \log\frac{1}{P_X(i)}\rceil$ bits.
	\begin{subex}[(1pt)]
	Construct the code for the probability distribution $P_X(1) = \frac{1}{2}$, $P_X(2) = \frac{1}{4}$, $P_X(3) = P_X(4) = \frac{1}{8}$
	\end{subex}
	\begin{subex}[(1pt)]
	Construct the code for the probability distribution $P_Y(1) = P_Y(2) = P_Y(3) = \frac{1}{3}$.
	\end{subex}
	\begin{subex}[(2pt)]
	Show that the Shannon code is prefix-free.
	\end{subex}
	\begin{subex}[(2pt)]
	Show that the average length $\ell_S$ of the Shannon code satisfies
	\[
	H(X) \leq \ell_S(P_X) < H(X) + 1.
	\]
	\end{subex}
\end{exercise}



\newcommand{\typsetA}{A^{(n)}_{\varepsilon}}
\newcommand{\typsetB}{B^{(n)}_{\delta}}

\begin{exercise}[Calculation of the typical set (8pt)]
To clarify the notion of a typical set $\typsetA$ and the smallest set of high probability $\typsetB$, we will calculate these sets for a simple example. Consider a sequence of i.i.d.\ binary random variables $X_1, X_2, . . . X_n$, where the probability that $P_X(1) = 0.6$ and $P_X(0) = 0.4$.
	\begin{subex}[(1pt)]
	Calculate $H(X)$.
	\end{subex}
	\begin{subex}[(3pt)]
	With $n = 25$ and $\varepsilon = 0.1$, which sequences fall in the typical set $\typsetA$? What is the probability of the typical set? How many elements are there in the typical set? (This involves computation of a table of probabilities for sequences with $k$ 1's, $0 \leq k \leq 25$, and finding those sequences that are in the typical set.)
\\\textbf{Hint:} Here is the table: \url{http://goo.gl/sQCPM0}
	\end{subex}
	\begin{subex}[(2pt)]
	How many elements are there in the smallest set that has probability 0.9? In other words, what is $|\typsetB|$ for $n = 25$ and $\delta = 0.1$?
	\end{subex}
	\begin{subex}[(2pt)]
	How many elements are there in the intersection $|\typsetA \cap \typsetB|$ of the sets computed in parts (b) and (c)? What is the probability of this intersection?
	\end{subex}
\end{exercise}

\begin{exercise}[Three random variables (3pt)]
Let $A,B,C$ be random variables such that
\begin{align*}
I(A;B) &= 0\\
I(A;C|B) &= I(A;B|C)\\
H(A|BC) &= 0
\end{align*}
Prove one of the following possible relations between $H(A)$ and $H(C)$:
\begin{align*}
H(A)&=H(C) \, ,\\
H(A) &\leq H(C) \, ,\\
H(A) &\geq H(C) \, ,\\
H(A) &< H(C) \, ,\\
H(A) &> H(C) \, .
\end{align*}
\end{exercise}

\pagebreak

\begin{exercise}[Piece of cake (2pt)]
A big cake is repeatedly sliced into two pieces. At every step, the \emph{smallest} part is discarded (or eaten). On the other part the process is continued. At every step, the remaining piece is cut randomly into two pieces with the following proportions:
\begin{align*}
\left(\frac{2}{3},\frac{1}{3}\right) &\mbox{ with probability } \frac{3}{4},\\
\left(\frac{2}{5},\frac{3}{5}\right) &\mbox{ with probability } \frac{1}{4}.\\
\end{align*}
For example, three consecutive cuts might result in a piece of cake of size $\frac{3}{5} \cdot \frac{2}{3} \cdot \frac{2}{3}$. Let $T_n$ be the fraction of the cake left after $n$ cuts. Describe the limiting behavior of $T_n$ as a function of $n$ when $n$ tends to infinity.

\textbf{Hint:} Let $C_i$ be a random variable describing the fraction of the cake that is cut (and kept) at the $i$th cut, i.e.\ $C_i = \frac{2}{3}$ with probability $\frac{3}{4}$, or $\frac{3}{5}$ otherwise.  Then use the weak law of large numbers.





\end{exercise}






\end{document}