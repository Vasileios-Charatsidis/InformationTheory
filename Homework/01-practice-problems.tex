\documentclass[a4paper,10pt,landscape,twocolumn]{scrartcl}

%% Settings
\newcommand\problemset{1}
\newcommand\deadline{Friday November 11th, 20:00h}
\newif\ifcomments
\commentsfalse % hide comments
%\commentstrue % show comments

% Packages
\usepackage[english]{exercises}
\usepackage{wasysym}
\usepackage{hyperref}
\hypersetup{colorlinks=true, urlcolor = blue, linkcolor = blue}

\begin{document}

\practiceproblems

{\sffamily\noindent
%This week's exercises deal with sets, counting and uniform probabilities.
This week's exercises deal with the basics of probability theory and entropy, as well as with different proof techniques. You do not have to hand in these exercises, they are for practicing only. Problems marked with a $\bigstar$ are generally a bit harder. If you have questions about any of the exercises, please post them in the \href{https://www.moodle.ch/lms/mod/forum/view.php?id=1761}{discussion forum on Moodle}, and try to help each other. We will also keep an eye on the forum.
%Your homework must be handed in \textbf{electronically via Moodle before \deadline}. This deadline is strict and late submissions are graded with a 0. At the end of the course, the lowest of your 7 weekly homework grades will be dropped. You are strongly encouraged to work together on the exercises, including the homework. However, after this discussion phase, you have to write down and submit your own individual solution. Numbers alone are never sufficient, always motivate your answers.
}

\enlargethispage{1cm}
\begin{exercise}[Two dice]
Consider an experiment where we throw two fair six-sided dice: a red one and a blue one.
	\begin{subex}
	What is the probability space $(\Omega,\mathcal{F},P)$ for this experiment? What would be the probability space if the dice were both red (i.e. indistinguishable)?
	\end{subex}
	\begin{subex}
	Let $X$ be the random variable that describes the sum of the two outcomes. Describe its range $\mathcal{X}$ and distribution $P_X$. What is $P_X(7) = P(X=7)$?
	\end{subex}
	\begin{subex}
	Let $Y$ be the random variable that describes the \emph{parity} of the sum, i.e. $\mathcal{Y} = \{\mathsf{even}, \mathsf{odd}\}$. What is $P_{X|Y}(7|\mathsf{odd})$? And $P_{X|Y}(7|\mathsf{even})$?
	\end{subex}
	\begin{subex}
	Verify that for an arbitrary random variable $X$, $(\mathcal{X},\mathcal{P}(X),P_X)$ is a probability space.
	\end{subex}
\end{exercise}

\begin{exercise}[Inverse probabilities]
What is the probability that two (or more) students in this exercise class have the same birthday? (Assume everybody was born in the same year.)
\end{exercise}

\begin{exercise}[Events]
Let $\mathcal{A}, \mathcal{B}$ be events (subsets of some sample space $\Omega$). Prove the following identities:
	\begin{subex}
	$P[\overline{\mathcal{A}}] = 1 - P[\mathcal{A}]$
	\end{subex}
	
	\begin{subex}
	$P[\mathcal{A} \cup \mathcal{B}] = P[\mathcal{A}] + P[\mathcal{B}] - P[\mathcal{A},\mathcal{B}]$
	\end{subex}
	
	\begin{subex}
	$P[\mathcal{A}] = P[\mathcal{A},\mathcal{B}] + P[\mathcal{A},\overline{\mathcal{B}}]$
	\end{subex}
\end{exercise}

\begin{exercise}[Proof by induction]

	\begin{subex}
	Prove by induction on $n$ that for all $n \in \mathbb{N}_+$,
	\[
	\sum_{i=1}^n i = \frac{n(n+1)}{2}.
	\]
	\end{subex}
	
	\begin{subex}[Union bound]
	Prove the union bound for a finite number of events, which states that for arbitrary events $\mathcal{A}_1, \mathcal{A}_2, ..., \mathcal{A}_n$,
	\[
	P \left( \bigcup_{i=1}^n \mathcal{A}_i \right) \leq \sum_{i=1}^n P(\mathcal{A}_i).
	\]
	\end{subex}

	\begin{subex**}
	Can you find an exact formula for $P\left( \bigcup_{i=1}^n \mathcal{A}_i \right)$?
	\end{subex**}

\end{exercise}


\enlargethispage{1cm}

\begin{exercise}[Properties of entropy]
Let $X$ and $Y$ be random variables.
	\begin{subex}
	Prove that $H(X) = 0$ if and only if $X$ is \emph{constant}, i.e. there is an $x_0 \in \mathcal{X}$ such that $P_X(x_0) = 1$, and $P_X(x') = 0$ for all $x' \neq x_0$.
	\end{subex}
	
	\begin{subex}
	Prove that $H(XY) = H(X) + H(Y)$ if and only if $X$ and $Y$ are independent.
	\end{subex}

	\begin{subex}
	Prove that $H(X) = \log |\mathcal{X}|$ if $X$ is uniformly distributed.
	\end{subex}

	\begin{subex**}
	Prove that $X$ is uniformly distributed if $H(X) = \log |\mathcal{X}|$.
	\end{subex**}
	
\end{exercise}

\begin{exercise}[Entropy of a deck of cards]
	\begin{subex}
	Compute the entropy of a perfectly shuffled deck of 52 cards (i.e.\ the set of cards is uniformly distributed over all possible orders).
	\end{subex}
	
	\begin{subex}
	Now suppose we have a perfectly shuffled big deck, consisting of two \emph{identical} decks of 52 cards (104 cards in total). You cannot tell the difference between, for example, the ace of spades of one deck and the ace of spades of the other. Compute the entropy of the shuffled big deck.
	\end{subex}
\end{exercise}


\end{document}