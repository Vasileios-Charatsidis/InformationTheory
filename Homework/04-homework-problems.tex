\documentclass[a4paper,10pt,landscape,twocolumn]{scrartcl}

%% Settings
\newcommand\problemset{3}
\newcommand\deadline{Friday December 2nd, 20:00h}
\newif\ifcomments
\commentsfalse % hide comments
%\commentstrue % show comments

% Packages
\usepackage[english]{exercises}
\usepackage{wasysym}
\usepackage{hyperref}
\hypersetup{colorlinks=true, urlcolor = blue, linkcolor = blue}

\usepackage{tikz}

\begin{document}

\homeworkproblems

{\sffamily\noindent
Your homework must be handed in \textbf{electronically via Moodle before \deadline}. This deadline is strict and late submissions are graded with a 0. At the end of the course, the lowest of your 6 weekly homework grades will be dropped. You are strongly encouraged to work together on the exercises, including the homework. However, after this discussion phase, you have to write down and submit your own individual solution. Numbers alone are never sufficient, always motivate your answers.
}


\begin{exercise}[Bottleneck (4pt)]
Suppose a Markov chain starts in one of $n$ states, necks down to $k < n$ states, and then fans back to $m > k$ states. Thus $X_1 \to X_2 \to X_3$, with $\mathcal{X}_1 = \{1, 2, ..., n\}$, $\mathcal{X}_2 = \{1, 2, ..., k\}$, and $\mathcal{X}_3 = \{1, 2, ..., m\}$.
\begin{subex}[(3pt)]
Show that the dependence of $X_1$ and $X_3$ is limited by the bottleneck by proving that $I(X_1;X_2) \leq \log k$.
\end{subex}
\begin{subex}[(1pt)]
Evaluate $I(X_1;X_3)$ for $k = 1$, and conclude that no dependence can survive such a bottleneck.
\end{subex}
\end{exercise}

\begin{exercise}[Run-length coding (6pt)]
Let $X_1, X_2, ..., X_n$ be (possibly dependent) binary random variables.
Suppose one calculates the run lengths $R = (R_1, R_2, ...)$ of this sequence (in order as they occur).
For example, the sequence $X = 0001100100$ yields run lengths $R = (3, 2, 2, 1, 2)$. Compare
$H(X_1, X_2, . . . , X_n)$, $H(R)$ and $H(X_n, R)$. Show all equalities and inequalities, and bound all the
differences.
\end{exercise}




\begin{exercise}
Stochastic processes / exam 2015
\end{exercise}

\begin{exercise}
Something with ECC
\end{exercise}







\end{document}